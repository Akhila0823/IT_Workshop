\documentclass[leqno]{article}
\usepackage{amsmath}
\usepackage[paperwidth=216mm,paperheight=279mm]{geometry}
\usepackage{amssymb}
\usepackage{graphicx}
\usepackage{hyperref}
\usepackage{xcolor}
\usepackage{amsthm}

\newtheorem{conjecture}{Conjecture}[section]

\newtheorem{theorem}{Theorem}[section]
\hypersetup{
	pdfborderstyle={/S/U/W 0} % underline of width 1pt
}


\title{\textbf{THE 18.821 MATHEMATICS PROJECT LAB REPORT
		[REPLACE THIS WITH YOUR OWN SHORT
		DESCRIPTIVE TITLE!]}}
\author{X. BURPS, P. GURPS}
\date{}
\makeatletter
\renewcommand\section{\@startsection {section}{1}{\z@}%
	{-3.5ex \@plus -1ex \@minus -.2ex}%
	{2.3ex \@plus.2ex}%
	{\normalfont\Large}}
\makeatother
\begin{document}
    \maketitle
ABSTRACT. This is a \LaTeX{} template for 18.821, which you can
use for your own reports.
\vspace{1cm}
\centering
\section{INTRODUCTION}
This brief document shows some examples of the use of \LaTeX{} and\par{}
indicates some special features of the Math Lab report style. The\par{} 
\href{http://stellar.mit.edu/S/course/18/sp13/18.821/}{\textcolor{blue}{\underline{course website}}} contains links to several \LaTeX{} manuals.\par{}
End the introduction by describing the contents of the paper sec­ \par{}
tion by section, and which team member(s) wrote each of them. For \par{}
instance, Section 6 discusses referencing, and is written by P. Gurps.
\section{\LaTeX{} Examples}
Here are some ways of producing mathematical symbols. Some are
pre-defined either in \LaTeX{} or in the AMS package which this document
loads. For instance, sums and integrals $\sum_{i=1}^{n} 1=n$ $\int_{0}^{n} x \, dx = \frac{n^2}{2}$ We’ve defined a few other symbols at the start of the document, for
instance $\mathbb{N,Q,Z,R}$ You can make marginal notes for yourself or your
co-authors like this:Unfinished here?
If you want to typeset equations, there are many choices, with or
without numbering:
\newline
\[
\int_{0}^{1} x \,dx=\frac{1}{2}
\]
\raggedright
or
\[
\sum_{i=1}^{\infty}=- \frac{1}{12}
 \]
\raggedright
or 
\[
1-1+1-\ldots=\frac{1}{2}
\]
\newpage
\centering
\section*{X. BURPS, P. GURPS}

\begin{figure}[h]
	\centering
	\includegraphics[width=0.7\linewidth]{pp.png}
	\caption{My first .pdf figure.}
	\label{fig:pp}
\end{figure}
If you want a number for an equation, do it like this:
\begin{equation}
\lim_{n \to \infty} \sum_{k=1}^{n} \frac{1}{k^2} =\frac{\pi}{6} .
\end{equation}
This can then be referred to as (1), which is much easier than keeping
track of numbers by hand. To group several equations, aligning on the
= sign, do it like this:

\begin{align*}
x + 2x^2 + 3x^3 &= 7 \\
y &= mx + c \\
&= 4x - 9.
\end{align*}
You can easily embed hyperlinks into the output .pdf document:
\href{http://stellar.mit.edu/S/course/18/sp13/18.821/}{\textcolor{blue}{\underline{click here}}} for example.
\section{IMAGES}
Figure 1 is an example of a .pdf image put into a floating environ­
ment, which means LaTeX will draw it wherever there’s enough space
left in your manuscript. Look at the .tex original to see how to insert
a figure like this.
\section{THEOREMS AND SUCH}
An example of a “conjecture environment” is given below, in Con­
jecture 4.1. Theorems, lemmas, propositions, definitions, and such all
use the same command with the appropriate name changed. In fact,
\section*{THE 18.821 REPORT}
if you look at the top of this .tex file, you can see where we’ve defined
these environments.
\begin{conjecture}[Vaught’s Conjecture]
Let $T$ be a countable com­
plete theory. If $T$ has fewer than $2^{\aleph_0}$  many countable models (up to
isomorphism), then it has countably many countable models.
\end{conjecture}
\begin{theorem} 
 When it rains it pours.
\end{theorem}
\begin{proof}
 Well, yes.
\end{proof}
\section{ Filetypes used by LaTeX}
	You will write your text as a .tex file using any text editor (though
	WYSIWYG editors are troublesome). Traditionally one then runs
	L A TEX and obtains a .dvi file, which can be viewed on the screen using a
	dvi viewer. To include images, and then prepare the file for printing or
	submission, one typically translates the .dvi into either .ps (Postscript)
	or .pdf (Adobe PDF).
	Your report will be submitted as a .pdf document. The pdflatex
	command produces a .pdf file directly from a .tex file. This command
	works well with included .pdf files, but does not handle .eps files.
	An .eps file can be converted to a .pdf file by viewing it and saving
	as a .pdf file, or by ps2pdf filename.eps, which produces
	filename.pdf. Under MikTeX with WinEdt, all necessary commands
	will appear under “Accessories” in the WinEdt menu.
	Finally, Matlab can be made to produce .eps files by typing
	print -deps filename
	at the prompt.
	\section{Quoting sources}

	In your work, keep notes of the literature you’ve used, including
	websites. Cite the references you use; failure to do so constitutes pla­
	giarism. Every bibliography item should be referenced somewhere in
	the paper. Quote as precisely as possible: [1, pages 76–78] rather than
	[1]. [2] was a useful background reference, too.\
 \cite{burps2008,gurps2008}
 \bibliographystyle{plain}
 \bibliography{reference.bib}
\section*{APPENDIX}
Appendices are useful for putting in code or data.	
\newpage
\raggedright
MIT OpenCourseWare
\href{http://ocw.mit.edu}{\textcolor{blue}{\underline{http://ocw.mit.edu}}}
\vspace{2cm}
\newline

18.821 Project Laboratory in Mathematics
\newline
Spring 2013
\vspace{2cm}
\newline
For information about citing these materials or our Terms of Use, visit: \href{http://ocw.mit.edu/terms.}{\textcolor{blue}{\underline{http://ocw.mit.edu/terms.}}}


\end{document}
